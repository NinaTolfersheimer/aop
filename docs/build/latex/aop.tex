%% Generated by Sphinx.
\def\sphinxdocclass{report}
\documentclass[letterpaper,10pt,english]{sphinxmanual}
\ifdefined\pdfpxdimen
   \let\sphinxpxdimen\pdfpxdimen\else\newdimen\sphinxpxdimen
\fi \sphinxpxdimen=.75bp\relax
\ifdefined\pdfimageresolution
    \pdfimageresolution= \numexpr \dimexpr1in\relax/\sphinxpxdimen\relax
\fi
%% let collapsible pdf bookmarks panel have high depth per default
\PassOptionsToPackage{bookmarksdepth=5}{hyperref}

\PassOptionsToPackage{booktabs}{sphinx}
\PassOptionsToPackage{colorrows}{sphinx}

\PassOptionsToPackage{warn}{textcomp}
\usepackage[utf8]{inputenc}
\ifdefined\DeclareUnicodeCharacter
% support both utf8 and utf8x syntaxes
  \ifdefined\DeclareUnicodeCharacterAsOptional
    \def\sphinxDUC#1{\DeclareUnicodeCharacter{"#1}}
  \else
    \let\sphinxDUC\DeclareUnicodeCharacter
  \fi
  \sphinxDUC{00A0}{\nobreakspace}
  \sphinxDUC{2500}{\sphinxunichar{2500}}
  \sphinxDUC{2502}{\sphinxunichar{2502}}
  \sphinxDUC{2514}{\sphinxunichar{2514}}
  \sphinxDUC{251C}{\sphinxunichar{251C}}
  \sphinxDUC{2572}{\textbackslash}
\fi
\usepackage{cmap}
\usepackage[T1]{fontenc}
\usepackage{amsmath,amssymb,amstext}
\usepackage{babel}



\usepackage{tgtermes}
\usepackage{tgheros}
\renewcommand{\ttdefault}{txtt}



\usepackage[Bjarne]{fncychap}
\usepackage{sphinx}

\fvset{fontsize=auto}
\usepackage{geometry}


% Include hyperref last.
\usepackage{hyperref}
% Fix anchor placement for figures with captions.
\usepackage{hypcap}% it must be loaded after hyperref.
% Set up styles of URL: it should be placed after hyperref.
\urlstyle{same}

\addto\captionsenglish{\renewcommand{\contentsname}{Contents:}}

\usepackage{sphinxmessages}
\setcounter{tocdepth}{1}



\title{aop}
\date{May 29, 2023}
\release{1.0}
\author{Amélie Solveigh Hohe}
\newcommand{\sphinxlogo}{\vbox{}}
\renewcommand{\releasename}{Release}
\makeindex
\begin{document}

\ifdefined\shorthandoff
  \ifnum\catcode`\=\string=\active\shorthandoff{=}\fi
  \ifnum\catcode`\"=\active\shorthandoff{"}\fi
\fi

\pagestyle{empty}
\sphinxmaketitle
\pagestyle{plain}
\sphinxtableofcontents
\pagestyle{normal}
\phantomsection\label{\detokenize{index::doc}}


\sphinxAtStartPar
\sphinxstylestrong{aop} is a Python module for astrophotographers and amateur astronomers that
creates observation logs compliant with \sphinxhref{https://tolfersheim.ddns.net/index.php/s/GabziFRsMD7FLeY}{the aop standard}. It offers \sphinxstyleemphasis{powerful}, yet \sphinxstyleemphasis{simple} and
\sphinxstyleemphasis{easy to understand} observation logs.

\begin{sphinxadmonition}{note}{Note:}
\sphinxAtStartPar
This project is under active development.
\end{sphinxadmonition}

\sphinxstepscope


\chapter{Usage}
\label{\detokenize{usage:usage}}\label{\detokenize{usage::doc}}

\section{Installation}
\label{\detokenize{usage:installation}}
\sphinxAtStartPar
To use aop, you first have to install it. Since it is a Python module, you
obviously have to have Python installed. Get it from \sphinxhref{https://www.python.org/download/}{the official website} if you don’t have it installed already.
You will also need the pip tool, but it usually ships with the Python
interpreter available over the link above.

\sphinxAtStartPar
After you have Python installed, we can shift our attention towards aop. Always
install it from source using pip. First download the source code from
\sphinxhref{https://github.com/NinaTolfersheimer/aop}{GitHub}.

\sphinxAtStartPar
After that, open a command\sphinxhyphen{}line or terminal and navigate to the folder you
stored the source code in, using

\begin{sphinxVerbatim}[commandchars=\\\{\}]
\PYG{n+nb}{cd }\PYG{n}{path}\PYG{p}{/}\PYG{n}{to}\PYG{p}{/}\PYG{n}{source}\PYG{p}{/}\PYG{n}{code}
\end{sphinxVerbatim}

\sphinxAtStartPar
We’ll provide a little example to clarify what is meant by that. Assume you’ve
stored the contents of the GitHub repository to
\sphinxcode{\sphinxupquote{C:\textbackslash{}\textbackslash{}Users\textbackslash{}\textbackslash{}Amelie\textbackslash{}\textbackslash{}Downloads\textbackslash{}\textbackslash{}python\_source\_code}} on your Windows machine. The
command\sphinxhyphen{}line usually starts out in your home directory. On Windows, it would
look like this:

\begin{sphinxVerbatim}[commandchars=\\\{\}]
\PYG{n}{C}\PYG{p}{:}\PYG{p}{\PYGZbs{}}\PYG{p}{\PYGZbs{}}\PYG{n}{Users}\PYG{p}{\PYGZbs{}}\PYG{p}{\PYGZbs{}}\PYG{n}{Amelie}\PYG{p}{\PYGZgt{}}\PYG{n+nb}{cd }\PYG{n}{Downloads}\PYG{p}{\PYGZbs{}}\PYG{p}{\PYGZbs{}}\PYG{n}{python\PYGZus{}source\PYGZus{}code}
\end{sphinxVerbatim}

\sphinxAtStartPar
By the way, the command\sphinxhyphen{}line on Windows is kind of hidden; to enter it open the
start menu and search for ‘cmd.exe’.

\sphinxAtStartPar
After you’ve moved to the correct directory, check it’s content with the
\sphinxcode{\sphinxupquote{dir}}\sphinxhyphen{}command. You should be able to see a sub\sphinxhyphen{}directory named ‘aop’ and a file
named ‘setup.py’. If not, inspect the directories above and below until you
found such a directory. The way you go to the parent directory, regardless of
your operating system, is

\begin{sphinxVerbatim}[commandchars=\\\{\}]
\PYG{n+nb}{cd }\PYG{p}{.}\PYG{p}{.}
\end{sphinxVerbatim}

\sphinxAtStartPar
On UNIX\sphinxhyphen{}like systems, like Linux and MacOS X, the command\sphinxhyphen{}line is usually
called ‘Terminal’ and a bit easier to find. Here, the path would be
\sphinxcode{\sphinxupquote{/home/Amelie/Downloads/python\_source\_code}} and the terminal would likely look
something like this:

\begin{sphinxVerbatim}[commandchars=\\\{\}]
\PYG{g+gp}{amelie\PYGZbs{}@amelies\PYGZhy{}computer:\PYGZti{}\PYGZdl{} }\PYG{n+nb}{cd}\PYG{+w}{ }Downloads/python\PYGZus{}source\PYGZus{}code
\end{sphinxVerbatim}

\sphinxAtStartPar
Now inspect it’s content with the command \sphinxcode{\sphinxupquote{ls}} and check whether you’re in the
right spot as described above.

\sphinxAtStartPar
If you are sure that you are in the correct source code directory, execute the
following command:

\begin{sphinxVerbatim}[commandchars=\\\{\}]
\PYG{n}{pip} \PYG{n}{install} \PYG{p}{.}
\end{sphinxVerbatim}

\sphinxAtStartPar
This should work regardless of your operating system, but if it doesn’t and you
are on Linux or MacOS, try

\begin{sphinxVerbatim}[commandchars=\\\{\}]
\PYG{g+gp}{\PYGZdl{} }pip3\PYG{+w}{ }install\PYG{+w}{ }.
\end{sphinxVerbatim}

\sphinxAtStartPar
instead.


\section{Interrupting a session}
\label{\detokenize{usage:interrupting-a-session}}
\sphinxAtStartPar
To interrupt a running session, you can use the \sphinxcode{\sphinxupquote{aop.Session.interrupt()}}
method:
\index{interrupt() (aop.Session method)@\spxentry{interrupt()}\spxextra{aop.Session method}}

\begin{fulllineitems}
\phantomsection\label{\detokenize{usage:aop.Session.interrupt}}
\pysigstartsignatures
\pysiglinewithargsret{\sphinxcode{\sphinxupquote{aop.Session.}}\sphinxbfcode{\sphinxupquote{interrupt}}}{\sphinxparam{\DUrole{n,n}{self}}, \sphinxparam{\DUrole{n,n}{time}\DUrole{o,o}{=}\DUrole{default_value}{\sphinxhyphen{}1}}}{}
\pysigstopsignatures
\sphinxAtStartPar
This method interrupts the session.

\sphinxAtStartPar
It sets the Session’s ‘interrupted’ flag to True in the instance
attribute as well as the parameter attribute. After this, it writes a
Session Event “SEEV SESSION INTERRUPTED” to the protocol and updates
the .aol parameter log.
\begin{quote}\begin{description}
\sphinxlineitem{Parameters}
\sphinxAtStartPar
\sphinxstyleliteralstrong{\sphinxupquote{time}} (\sphinxstyleliteralemphasis{\sphinxupquote{str}}) \textendash{} An ISO 8601 conform string of the UTC datetime you want your
observation to be interrupted at. The default is \sphinxhyphen{}1, in which case
the current UTC datetime will be used.

\sphinxlineitem{Raises}
\sphinxAtStartPar
{\hyperref[\detokenize{usage:aop.AolNotFoundError.AolNotFoundError}]{\sphinxcrossref{\sphinxstyleliteralstrong{\sphinxupquote{aop.AolNotFoundError.AolNotFoundError}}}}} \textendash{} If the .aol file is not found.

\end{description}\end{quote}

\end{fulllineitems}

\index{aop.AolNotFoundError.AolNotFoundError@\spxentry{aop.AolNotFoundError.AolNotFoundError}}

\begin{fulllineitems}
\phantomsection\label{\detokenize{usage:aop.AolNotFoundError.AolNotFoundError}}
\pysigstartsignatures
\pysigline{\sphinxbfcode{\sphinxupquote{exception\DUrole{w,w}{  }}}\sphinxcode{\sphinxupquote{aop.AolNotFoundError.}}\sphinxbfcode{\sphinxupquote{AolNotFoundError}}}
\pysigline{\sphinxbfcode{\sphinxupquote{Raised~upon~trying~to~interact~with~a~non\sphinxhyphen{}existing~.aol~file.}}}
\pysigstopsignatures
\end{fulllineitems}



\section{Starting a session}
\label{\detokenize{usage:starting-a-session}}
\sphinxstepscope


\chapter{aop package}
\label{\detokenize{aop:aop-package}}\label{\detokenize{aop::doc}}

\section{Submodules}
\label{\detokenize{aop:submodules}}

\section{aop.aop module}
\label{\detokenize{aop:aop-aop-module}}

\section{aop.tools module}
\label{\detokenize{aop:aop-tools-module}}

\section{Module contents}
\label{\detokenize{aop:module-contents}}
\sphinxstepscope


\chapter{aop}
\label{\detokenize{modules:aop}}\label{\detokenize{modules::doc}}
\sphinxstepscope


\chapter{API}
\label{\detokenize{api:api}}\label{\detokenize{api::doc}}

\begin{savenotes}\sphinxattablestart
\sphinxthistablewithglobalstyle
\sphinxthistablewithnovlinesstyle
\centering
\begin{tabulary}{\linewidth}[t]{\X{1}{2}\X{1}{2}}
\sphinxtoprule
\sphinxtableatstartofbodyhook\sphinxbottomrule
\end{tabulary}
\sphinxtableafterendhook\par
\sphinxattableend\end{savenotes}

\sphinxstepscope


\chapter{API Reference}
\label{\detokenize{autoapi/index:api-reference}}\label{\detokenize{autoapi/index::doc}}
\sphinxAtStartPar
This page contains auto\sphinxhyphen{}generated API reference documentation %
\begin{footnote}[1]\sphinxAtStartFootnote
Created with \sphinxhref{https://github.com/readthedocs/sphinx-autoapi}{sphinx\sphinxhyphen{}autoapi}
%
\end{footnote}.

\sphinxstepscope


\section{\sphinxstyleliteralintitle{\sphinxupquote{aop}}}
\label{\detokenize{autoapi/aop/index:module-aop}}\label{\detokenize{autoapi/aop/index:aop}}\label{\detokenize{autoapi/aop/index::doc}}\index{module@\spxentry{module}!aop@\spxentry{aop}}\index{aop@\spxentry{aop}!module@\spxentry{module}}
\sphinxAtStartPar
@author: Amélie Solveigh Hohe

\sphinxAtStartPar
About: This package provides the background functionality for an implementation of
the Astronomical Observation Protocol standard v1.0 (aop). It fully implements the
standard, but is meant to be implemented by a front\sphinxhyphen{}end app.

\sphinxAtStartPar
Third\sphinxhyphen{}party dependencies are listed in requirements.txt.


\subsection{Submodules}
\label{\detokenize{autoapi/aop/index:submodules}}
\sphinxstepscope


\subsubsection{\sphinxstyleliteralintitle{\sphinxupquote{aop.aop}}}
\label{\detokenize{autoapi/aop/aop/index:module-aop.aop}}\label{\detokenize{autoapi/aop/aop/index:aop-aop}}\label{\detokenize{autoapi/aop/aop/index::doc}}\index{module@\spxentry{module}!aop.aop@\spxentry{aop.aop}}\index{aop.aop@\spxentry{aop.aop}!module@\spxentry{module}}
\sphinxAtStartPar
@author: Amélie Solveigh Hohe

\sphinxAtStartPar
This module contains the main classes and functions of the aop package.


\paragraph{Module Contents}
\label{\detokenize{autoapi/aop/aop/index:module-contents}}

\subparagraph{Classes}
\label{\detokenize{autoapi/aop/aop/index:classes}}

\begin{savenotes}\sphinxattablestart
\sphinxthistablewithglobalstyle
\sphinxthistablewithnovlinesstyle
\centering
\begin{tabulary}{\linewidth}[t]{\X{1}{2}\X{1}{2}}
\sphinxtoprule
\sphinxtableatstartofbodyhook
\sphinxAtStartPar
{\hyperref[\detokenize{autoapi/aop/aop/index:aop.aop.Session}]{\sphinxcrossref{\sphinxcode{\sphinxupquote{Session}}}}}
&
\sphinxAtStartPar
A class representing an astronomical observing session.
\\
\sphinxbottomrule
\end{tabulary}
\sphinxtableafterendhook\par
\sphinxattableend\end{savenotes}


\subparagraph{Functions}
\label{\detokenize{autoapi/aop/aop/index:functions}}

\begin{savenotes}\sphinxattablestart
\sphinxthistablewithglobalstyle
\sphinxthistablewithnovlinesstyle
\centering
\begin{tabulary}{\linewidth}[t]{\X{1}{2}\X{1}{2}}
\sphinxtoprule
\sphinxtableatstartofbodyhook
\sphinxAtStartPar
{\hyperref[\detokenize{autoapi/aop/aop/index:aop.aop.current_jd}]{\sphinxcrossref{\sphinxcode{\sphinxupquote{current\_jd}}}}}(\(\rightarrow\) numpy.float64)
&
\sphinxAtStartPar
Returns the Julian Date for the current UTC or a custom datetime.
\\
\sphinxhline
\sphinxAtStartPar
{\hyperref[\detokenize{autoapi/aop/aop/index:aop.aop.generate_observation_id}]{\sphinxcrossref{\sphinxcode{\sphinxupquote{generate\_observation\_id}}}}}(\(\rightarrow\) str)
&
\sphinxAtStartPar
This function generates a unique observation ID.
\\
\sphinxhline
\sphinxAtStartPar
{\hyperref[\detokenize{autoapi/aop/aop/index:aop.aop.create_entry_id}]{\sphinxcrossref{\sphinxcode{\sphinxupquote{create\_entry\_id}}}}}(\(\rightarrow\) str)
&
\sphinxAtStartPar
Creates a unique identifier for each and every entry in an .aop protocol.
\\
\sphinxhline
\sphinxAtStartPar
{\hyperref[\detokenize{autoapi/aop/aop/index:aop.aop.parse_session}]{\sphinxcrossref{\sphinxcode{\sphinxupquote{parse\_session}}}}}(\(\rightarrow\) Session)
&
\sphinxAtStartPar
This function parses a session from memory to a new Session object.
\\
\sphinxbottomrule
\end{tabulary}
\sphinxtableafterendhook\par
\sphinxattableend\end{savenotes}
\index{current\_jd() (in module aop.aop)@\spxentry{current\_jd()}\spxextra{in module aop.aop}}

\begin{fulllineitems}
\phantomsection\label{\detokenize{autoapi/aop/aop/index:aop.aop.current_jd}}
\pysigstartsignatures
\pysiglinewithargsret{\sphinxcode{\sphinxupquote{aop.aop.}}\sphinxbfcode{\sphinxupquote{current\_jd}}}{\sphinxparam{\DUrole{n,n}{time}\DUrole{p,p}{:}\DUrole{w,w}{  }\DUrole{n,n}{str}\DUrole{w,w}{  }\DUrole{o,o}{=}\DUrole{w,w}{  }\DUrole{default_value}{\textquotesingle{}current\textquotesingle{}}}}{{ $\rightarrow$ numpy.float64}}
\pysigstopsignatures
\sphinxAtStartPar
Returns the Julian Date for the current UTC or a custom datetime.

\sphinxAtStartPar
It makes use of astropy’s \sphinxcode{\sphinxupquote{Time}} class to represent the datetime given as
a Julian Date.
\begin{quote}\begin{description}
\sphinxlineitem{Parameters}
\sphinxAtStartPar
\sphinxstyleliteralstrong{\sphinxupquote{time}} (\sphinxcode{\sphinxupquote{str}}, optional) \textendash{} An ISO 8601 conform string of the UTC datetime you want to be converted
to a Julian Date. If \sphinxcode{\sphinxupquote{time}} is “current”, the current UTC
datetime will be used, defaults to “current”.

\sphinxlineitem{Raises}
\sphinxAtStartPar
\sphinxstyleliteralstrong{\sphinxupquote{TypeError}} \textendash{} If the \sphinxcode{\sphinxupquote{time}} argument is not of type \sphinxcode{\sphinxupquote{str}}.

\end{description}\end{quote}
\begin{description}
\sphinxlineitem{:raises \sphinxcode{\sphinxupquote{InvalidTimeStringError}}: If the \sphinxcode{\sphinxupquote{time}} argument is of type \sphinxcode{\sphinxupquote{str}} but not interpretable as}
\sphinxAtStartPar
representing a time to astropy.time.Time.

\end{description}
\begin{quote}\begin{description}
\sphinxlineitem{Returns}
\sphinxAtStartPar
The Julian Date corresponding to the datetime provided.

\sphinxlineitem{Return type}
\sphinxAtStartPar
\sphinxcode{\sphinxupquote{numpy.float64}}

\end{description}\end{quote}

\end{fulllineitems}

\index{generate\_observation\_id() (in module aop.aop)@\spxentry{generate\_observation\_id()}\spxextra{in module aop.aop}}

\begin{fulllineitems}
\phantomsection\label{\detokenize{autoapi/aop/aop/index:aop.aop.generate_observation_id}}
\pysigstartsignatures
\pysiglinewithargsret{\sphinxcode{\sphinxupquote{aop.aop.}}\sphinxbfcode{\sphinxupquote{generate\_observation\_id}}}{\sphinxparam{\DUrole{n,n}{digits}\DUrole{p,p}{:}\DUrole{w,w}{  }\DUrole{n,n}{int}\DUrole{w,w}{  }\DUrole{o,o}{=}\DUrole{w,w}{  }\DUrole{default_value}{10}}}{{ $\rightarrow$ str}}
\pysigstopsignatures
\sphinxAtStartPar
This function generates a unique observation ID.

\sphinxAtStartPar
The ID is generated as such: YYYY\sphinxhyphen{}mm\sphinxhyphen{}dd\sphinxhyphen{}HH\sphinxhyphen{}MM\sphinxhyphen{}SS\sphinxhyphen{}uuuuuuuuuu,
where:
\begin{itemize}
\item {} 
\sphinxAtStartPar
YYYY: current UTC year

\item {} 
\sphinxAtStartPar
mm: current UTC month

\item {} 
\sphinxAtStartPar
dd: current UTC day

\item {} 
\sphinxAtStartPar
HH: current UTC hour

\item {} 
\sphinxAtStartPar
MM: current UTC minute

\item {} 
\sphinxAtStartPar
SS: current UTC second

\item {} 
\sphinxAtStartPar
uuuuuuuuuu: a \sphinxcode{\sphinxupquote{digits}}\sphinxhyphen{}long unique identifier (10 digits per default)

\end{itemize}
\begin{quote}\begin{description}
\sphinxlineitem{Parameters}
\sphinxAtStartPar
\sphinxstyleliteralstrong{\sphinxupquote{digits}} (\sphinxcode{\sphinxupquote{int}}, optional) \textendash{} The number of digits to be used for the unique identifier part of
the observation ID, defaults to 10.

\sphinxlineitem{Returns}
\sphinxAtStartPar
The generated observation ID.

\sphinxlineitem{Return type}
\sphinxAtStartPar
\sphinxcode{\sphinxupquote{str}}

\end{description}\end{quote}

\end{fulllineitems}

\index{create\_entry\_id() (in module aop.aop)@\spxentry{create\_entry\_id()}\spxextra{in module aop.aop}}

\begin{fulllineitems}
\phantomsection\label{\detokenize{autoapi/aop/aop/index:aop.aop.create_entry_id}}
\pysigstartsignatures
\pysiglinewithargsret{\sphinxcode{\sphinxupquote{aop.aop.}}\sphinxbfcode{\sphinxupquote{create\_entry\_id}}}{\sphinxparam{\DUrole{n,n}{time}\DUrole{p,p}{:}\DUrole{w,w}{  }\DUrole{n,n}{str}\DUrole{w,w}{  }\DUrole{o,o}{=}\DUrole{w,w}{  }\DUrole{default_value}{\textquotesingle{}current\textquotesingle{}}}, \sphinxparam{\DUrole{n,n}{digits}\DUrole{p,p}{:}\DUrole{w,w}{  }\DUrole{n,n}{int}\DUrole{w,w}{  }\DUrole{o,o}{=}\DUrole{w,w}{  }\DUrole{default_value}{30}}}{{ $\rightarrow$ str}}
\pysigstopsignatures
\sphinxAtStartPar
Creates a unique identifier for each and every entry in an .aop protocol.
This identifier is unique even across observations.
\begin{quote}\begin{description}
\sphinxlineitem{Parameters}\begin{itemize}
\item {} 
\sphinxAtStartPar
\sphinxstyleliteralstrong{\sphinxupquote{time}} (\sphinxcode{\sphinxupquote{str}}, optional) \textendash{} If equal to “current”, the current UTC datetime is used for
entry ID creation. You can also pass an ISO 8601 conform string to
time, if the time of the entry is not the current time this method is
called, defaults to “current”.

\item {} 
\sphinxAtStartPar
\sphinxstyleliteralstrong{\sphinxupquote{digits}} (\sphinxcode{\sphinxupquote{int}}, optional) \textendash{} The number of digits to use for the unique part of the entry ID, defaults to 30.

\end{itemize}

\sphinxlineitem{Raises}
\sphinxAtStartPar
\sphinxstyleliteralstrong{\sphinxupquote{TypeError}} \textendash{} If time is not a string.

\end{description}\end{quote}
\begin{description}
\sphinxlineitem{:raises \sphinxcode{\sphinxupquote{InvalidTimeStringError}}: If a string different from “current” is provided as \sphinxcode{\sphinxupquote{time}} argument, but}
\sphinxAtStartPar
it is not ISO 8601 conform and therefore does not constitute a valid time string.

\end{description}
\begin{quote}\begin{description}
\sphinxlineitem{Returns}
\sphinxAtStartPar

\sphinxAtStartPar
The entry ID generated. It follows the syntax YYYYMMDDhhmmssffffff\sphinxhyphen{}u,
where:
\begin{itemize}
\item {} 
\sphinxAtStartPar
YYYY   is the specified UTC year,

\item {} 
\sphinxAtStartPar
MM     is the specified UTC month,

\item {} 
\sphinxAtStartPar
DD     is the specified UTC day,

\item {} 
\sphinxAtStartPar
hh     is the specified UTC hour,

\item {} 
\sphinxAtStartPar
mm     is the specified UTC month,

\item {} 
\sphinxAtStartPar
ss     is the specified UTC second,

\item {} 
\sphinxAtStartPar
ffffff is the specified fraction of a UTC second and

\item {} 
\sphinxAtStartPar
u      represents ‘digits’ of unique identifier characters.

\end{itemize}


\sphinxlineitem{Return type}
\sphinxAtStartPar
\sphinxcode{\sphinxupquote{str}}

\end{description}\end{quote}

\end{fulllineitems}

\index{Session (class in aop.aop)@\spxentry{Session}\spxextra{class in aop.aop}}

\begin{fulllineitems}
\phantomsection\label{\detokenize{autoapi/aop/aop/index:aop.aop.Session}}
\pysigstartsignatures
\pysiglinewithargsret{\sphinxbfcode{\sphinxupquote{class\DUrole{w,w}{  }}}\sphinxcode{\sphinxupquote{aop.aop.}}\sphinxbfcode{\sphinxupquote{Session}}}{\sphinxparam{\DUrole{n,n}{filepath}\DUrole{p,p}{:}\DUrole{w,w}{  }\DUrole{n,n}{str}}, \sphinxparam{\DUrole{o,o}{**}\DUrole{n,n}{kwargs}}}{}
\pysigstopsignatures
\sphinxAtStartPar
A class representing an astronomical observing session.

\sphinxAtStartPar
The Session class provides several public methods representing different
actions and events that occur throughout an astronomical observation.
It is logged according to the Astronomical Observation Protocol Standard v1.0.
\index{obsID (aop.aop.Session attribute)@\spxentry{obsID}\spxextra{aop.aop.Session attribute}}

\begin{fulllineitems}
\phantomsection\label{\detokenize{autoapi/aop/aop/index:aop.aop.Session.obsID}}
\pysigstartsignatures
\pysigline{\sphinxbfcode{\sphinxupquote{obsID}}}
\pysigstopsignatures
\sphinxAtStartPar
The unique observation ID generated using the {\color{red}\bfseries{}:function:\textasciigrave{}generate\_observation\_id()\textasciigrave{}}
function.

\end{fulllineitems}

\index{filepath (aop.aop.Session attribute)@\spxentry{filepath}\spxextra{aop.aop.Session attribute}}

\begin{fulllineitems}
\phantomsection\label{\detokenize{autoapi/aop/aop/index:aop.aop.Session.filepath}}
\pysigstartsignatures
\pysigline{\sphinxbfcode{\sphinxupquote{filepath}}}
\pysigstopsignatures
\sphinxAtStartPar
The path where the implementing script wants aop to store its files. This could be a part of the implementing
script’s installation directory, for example.

\end{fulllineitems}

\index{state (aop.aop.Session attribute)@\spxentry{state}\spxextra{aop.aop.Session attribute}}

\begin{fulllineitems}
\phantomsection\label{\detokenize{autoapi/aop/aop/index:aop.aop.Session.state}}
\pysigstartsignatures
\pysigline{\sphinxbfcode{\sphinxupquote{state}}}
\pysigstopsignatures
\sphinxAtStartPar
A status flag indicating the current status of the observing session.
The class methods set this flag to either
\begin{itemize}
\item {} 
\sphinxAtStartPar
“running”,

\item {} 
\sphinxAtStartPar
“aborted” or

\item {} 
\sphinxAtStartPar
“ended”.

\end{itemize}

\sphinxAtStartPar
Initialized in \sphinxcode{\sphinxupquote{\_\_init\_\_()}} to None, updated in \sphinxcode{\sphinxupquote{start()}} to “running”.

\end{fulllineitems}

\index{interrupted (aop.aop.Session attribute)@\spxentry{interrupted}\spxextra{aop.aop.Session attribute}}

\begin{fulllineitems}
\phantomsection\label{\detokenize{autoapi/aop/aop/index:aop.aop.Session.interrupted}}
\pysigstartsignatures
\pysigline{\sphinxbfcode{\sphinxupquote{interrupted}}\sphinxbfcode{\sphinxupquote{\DUrole{w,w}{  }\DUrole{p,p}{=}\DUrole{w,w}{  }False}}}
\pysigstopsignatures
\sphinxAtStartPar
A status flag indicating whether the session is currently interrupted.
Initialized as False.

\end{fulllineitems}

\index{conditionDescription (aop.aop.Session attribute)@\spxentry{conditionDescription}\spxextra{aop.aop.Session attribute}}

\begin{fulllineitems}
\phantomsection\label{\detokenize{autoapi/aop/aop/index:aop.aop.Session.conditionDescription}}
\pysigstartsignatures
\pysigline{\sphinxbfcode{\sphinxupquote{conditionDescription}}}
\pysigstopsignatures
\sphinxAtStartPar
A short description of the observing conditions.

\end{fulllineitems}

\index{temp (aop.aop.Session attribute)@\spxentry{temp}\spxextra{aop.aop.Session attribute}}

\begin{fulllineitems}
\phantomsection\label{\detokenize{autoapi/aop/aop/index:aop.aop.Session.temp}}
\pysigstartsignatures
\pysigline{\sphinxbfcode{\sphinxupquote{temp}}}
\pysigstopsignatures
\sphinxAtStartPar
The temperature at the observing site in °C.

\end{fulllineitems}

\index{pressure (aop.aop.Session attribute)@\spxentry{pressure}\spxextra{aop.aop.Session attribute}}

\begin{fulllineitems}
\phantomsection\label{\detokenize{autoapi/aop/aop/index:aop.aop.Session.pressure}}
\pysigstartsignatures
\pysigline{\sphinxbfcode{\sphinxupquote{pressure}}}
\pysigstopsignatures
\sphinxAtStartPar
The air pressure at the observing site in hPa.

\end{fulllineitems}

\index{humidity (aop.aop.Session attribute)@\spxentry{humidity}\spxextra{aop.aop.Session attribute}}

\begin{fulllineitems}
\phantomsection\label{\detokenize{autoapi/aop/aop/index:aop.aop.Session.humidity}}
\pysigstartsignatures
\pysigline{\sphinxbfcode{\sphinxupquote{humidity}}}
\pysigstopsignatures
\sphinxAtStartPar
The air humidity at the observing site in \%.

\end{fulllineitems}

\index{\_\_repr\_\_() (aop.aop.Session method)@\spxentry{\_\_repr\_\_()}\spxextra{aop.aop.Session method}}

\begin{fulllineitems}
\phantomsection\label{\detokenize{autoapi/aop/aop/index:aop.aop.Session.__repr__}}
\pysigstartsignatures
\pysiglinewithargsret{\sphinxbfcode{\sphinxupquote{\_\_repr\_\_}}}{}{{ $\rightarrow$ str}}
\pysigstopsignatures
\sphinxAtStartPar
A Session object is represented by its attributes.
\begin{quote}\begin{description}
\sphinxlineitem{Returns}
\sphinxAtStartPar
A string containing all the instance’s attributes and their values
in line format.

\sphinxlineitem{Return type}
\sphinxAtStartPar
\sphinxcode{\sphinxupquote{str}}

\end{description}\end{quote}

\end{fulllineitems}

\index{start() (aop.aop.Session method)@\spxentry{start()}\spxextra{aop.aop.Session method}}

\begin{fulllineitems}
\phantomsection\label{\detokenize{autoapi/aop/aop/index:aop.aop.Session.start}}
\pysigstartsignatures
\pysiglinewithargsret{\sphinxbfcode{\sphinxupquote{start}}}{\sphinxparam{\DUrole{n,n}{time}\DUrole{p,p}{:}\DUrole{w,w}{  }\DUrole{n,n}{str}\DUrole{w,w}{  }\DUrole{o,o}{=}\DUrole{w,w}{  }\DUrole{default_value}{\textquotesingle{}current\textquotesingle{}}}}{{ $\rightarrow$ None}}
\pysigstopsignatures
\sphinxAtStartPar
This method is called to start the observing session.

\sphinxAtStartPar
By not starting the observation when a Session object is created, it is
possible to prepare the Session object pre\sphinxhyphen{}observation as well as
parse existing protocols from memory into a new Session object. It
changes the Session’s “state” flag to “running” (attribute and
parameter), as well as generating an observation ID, setting up a
directory for the protocol and parameter log to live in, and writing
the initial files to that directory. It also writes a Session Event
“SEEV SESSION \%obsID\% STARTED” to .aop.
\begin{quote}\begin{description}
\sphinxlineitem{Parameters}
\sphinxAtStartPar
\sphinxstyleliteralstrong{\sphinxupquote{time}} (\sphinxcode{\sphinxupquote{str}}, optional) \textendash{} An ISO 8601 conform string of the UTC datetime you want your
observation to start. Can also be “current”, in which case the current
UTC datetime will be used, defaults to “current”.

\sphinxlineitem{Raises}
\sphinxAtStartPar
\sphinxstyleliteralstrong{\sphinxupquote{PermissionError}} \textendash{} If the user does not have the adequate access rights for reading from or writing to the
.aol or .aop file.

\end{description}\end{quote}

\end{fulllineitems}

\index{\_\_write\_to\_aop() (aop.aop.Session static method)@\spxentry{\_\_write\_to\_aop()}\spxextra{aop.aop.Session static method}}

\begin{fulllineitems}
\phantomsection\label{\detokenize{autoapi/aop/aop/index:aop.aop.Session.__write_to_aop}}
\pysigstartsignatures
\pysiglinewithargsret{\sphinxbfcode{\sphinxupquote{static\DUrole{w,w}{  }}}\sphinxbfcode{\sphinxupquote{\_\_write\_to\_aop}}}{\sphinxparam{\DUrole{n,n}{self}}, \sphinxparam{\DUrole{n,n}{opcode}\DUrole{p,p}{:}\DUrole{w,w}{  }\DUrole{n,n}{str}}, \sphinxparam{\DUrole{n,n}{argument}\DUrole{p,p}{:}\DUrole{w,w}{  }\DUrole{n,n}{str}}, \sphinxparam{\DUrole{n,n}{time}\DUrole{p,p}{:}\DUrole{w,w}{  }\DUrole{n,n}{str}\DUrole{w,w}{  }\DUrole{o,o}{=}\DUrole{w,w}{  }\DUrole{default_value}{\textquotesingle{}current\textquotesingle{}}}}{{ $\rightarrow$ None}}
\pysigstopsignatures
\sphinxAtStartPar
This pseudo\sphinxhyphen{}private method is called to update the .aop protocol file.

\sphinxAtStartPar
For the syntax, check with the Astronomical Observation Protocol Syntax
Guide.
\begin{quote}\begin{description}
\sphinxlineitem{Parameters}\begin{itemize}
\item {} 
\sphinxAtStartPar
\sphinxstyleliteralstrong{\sphinxupquote{opcode}} (\sphinxcode{\sphinxupquote{str}}) \textendash{} The operation code of the event to be written to protocol, as
described in the AOP Syntax Guide.

\item {} 
\sphinxAtStartPar
\sphinxstyleliteralstrong{\sphinxupquote{argument}} (\sphinxcode{\sphinxupquote{str}}) \textendash{} Whatever is to be written to the argument position in the .aop
protocol entry.

\item {} 
\sphinxAtStartPar
\sphinxstyleliteralstrong{\sphinxupquote{time}} (\sphinxcode{\sphinxupquote{str}}, optional) \textendash{} An ISO 8601 conform string of the UTC datetime you want to use. Can also be “current”, in which
case the current UTC datetime will be used.
In most cases, however, the calling method will pass its own time
argument on to \_\_write\_to\_aop(), defaults to “current”.

\end{itemize}

\sphinxlineitem{Raises}
\sphinxAtStartPar
\sphinxstyleliteralstrong{\sphinxupquote{PermissionError}} \textendash{} If the user does not have the adequate access rights for writing to the .aop file.

\end{description}\end{quote}

\end{fulllineitems}

\index{\_\_write\_to\_aol() (aop.aop.Session static method)@\spxentry{\_\_write\_to\_aol()}\spxextra{aop.aop.Session static method}}

\begin{fulllineitems}
\phantomsection\label{\detokenize{autoapi/aop/aop/index:aop.aop.Session.__write_to_aol}}
\pysigstartsignatures
\pysiglinewithargsret{\sphinxbfcode{\sphinxupquote{static\DUrole{w,w}{  }}}\sphinxbfcode{\sphinxupquote{\_\_write\_to\_aol}}}{\sphinxparam{\DUrole{n,n}{self}}, \sphinxparam{\DUrole{n,n}{parameter}\DUrole{p,p}{:}\DUrole{w,w}{  }\DUrole{n,n}{str}}, \sphinxparam{\DUrole{n,n}{assigned\_value}}}{{ $\rightarrow$ None}}
\pysigstopsignatures
\sphinxAtStartPar
This pseudo\sphinxhyphen{}private method is used to update the .aol parameter log.

\sphinxAtStartPar
It takes two arguments, the first being the parameter name being updated,
the second one being the value it is assigned.
\begin{quote}\begin{description}
\sphinxlineitem{Parameters}\begin{itemize}
\item {} 
\sphinxAtStartPar
\sphinxstyleliteralstrong{\sphinxupquote{parameter}} (\sphinxcode{\sphinxupquote{str}}) \textendash{} The name of the parameter being updated.

\item {} 
\sphinxAtStartPar
\sphinxstyleliteralstrong{\sphinxupquote{assigned\_value}} (\sphinxstyleliteralemphasis{\sphinxupquote{any}}) \textendash{} The value the parameter should be assigned. Typically, this
is a string or boolean.

\end{itemize}

\sphinxlineitem{Raises}
\sphinxAtStartPar
\sphinxstyleliteralstrong{\sphinxupquote{PermissionError}} \textendash{} If the user does not have the adequate access rights for writing to the .aol file.

\end{description}\end{quote}

\end{fulllineitems}

\index{interrupt() (aop.aop.Session method)@\spxentry{interrupt()}\spxextra{aop.aop.Session method}}

\begin{fulllineitems}
\phantomsection\label{\detokenize{autoapi/aop/aop/index:aop.aop.Session.interrupt}}
\pysigstartsignatures
\pysiglinewithargsret{\sphinxbfcode{\sphinxupquote{interrupt}}}{\sphinxparam{\DUrole{n,n}{time}\DUrole{p,p}{:}\DUrole{w,w}{  }\DUrole{n,n}{str}\DUrole{w,w}{  }\DUrole{o,o}{=}\DUrole{w,w}{  }\DUrole{default_value}{\textquotesingle{}current\textquotesingle{}}}}{{ $\rightarrow$ None}}
\pysigstopsignatures
\sphinxAtStartPar
This method interrupts the session.

\sphinxAtStartPar
It sets the Session’s \sphinxcode{\sphinxupquote{interrupted}} flag to True in the instance
attribute as well as the \sphinxcode{\sphinxupquote{parameter}} attribute. After this, it writes a
Session Event “SEEV SESSION INTERRUPTED” to the protocol and updates
the .aol parameter log.
\begin{quote}\begin{description}
\sphinxlineitem{Parameters}
\sphinxAtStartPar
\sphinxstyleliteralstrong{\sphinxupquote{time}} (\sphinxcode{\sphinxupquote{str}}, optional) \textendash{} An ISO 8601 conform string of the UTC datetime you want your
observation to be interrupted at. Can also be “current”, in which case
the current UTC datetime will be used, defaults to “current”.

\end{description}\end{quote}

\end{fulllineitems}

\index{resume() (aop.aop.Session method)@\spxentry{resume()}\spxextra{aop.aop.Session method}}

\begin{fulllineitems}
\phantomsection\label{\detokenize{autoapi/aop/aop/index:aop.aop.Session.resume}}
\pysigstartsignatures
\pysiglinewithargsret{\sphinxbfcode{\sphinxupquote{resume}}}{\sphinxparam{\DUrole{n,n}{time}\DUrole{p,p}{:}\DUrole{w,w}{  }\DUrole{n,n}{str}\DUrole{w,w}{  }\DUrole{o,o}{=}\DUrole{w,w}{  }\DUrole{default_value}{\textquotesingle{}current\textquotesingle{}}}}{{ $\rightarrow$ None}}
\pysigstopsignatures
\sphinxAtStartPar
This method resumes the session.

\sphinxAtStartPar
It sets the Session’s \sphinxcode{\sphinxupquote{interrupted}} flag to False in the instance
attribute as well as the \sphinxcode{\sphinxupquote{parameter}} attribute. After this, it writes a
Session Event “SEEV SESSION RESUMED” to the protocol and updates the
.aol parameter log.
\begin{quote}\begin{description}
\sphinxlineitem{Parameters}
\sphinxAtStartPar
\sphinxstyleliteralstrong{\sphinxupquote{time}} (\sphinxcode{\sphinxupquote{str}}, optional) \textendash{} An ISO 8601 conform string of the UTC datetime you want your
observation to be resumed at. Can also be “current”, in which case the
current UTC datetime will be used, defaults to “current”.

\end{description}\end{quote}

\end{fulllineitems}

\index{abort() (aop.aop.Session method)@\spxentry{abort()}\spxextra{aop.aop.Session method}}

\begin{fulllineitems}
\phantomsection\label{\detokenize{autoapi/aop/aop/index:aop.aop.Session.abort}}
\pysigstartsignatures
\pysiglinewithargsret{\sphinxbfcode{\sphinxupquote{abort}}}{\sphinxparam{\DUrole{n,n}{reason}\DUrole{p,p}{:}\DUrole{w,w}{  }\DUrole{n,n}{str}}, \sphinxparam{\DUrole{n,n}{time}\DUrole{p,p}{:}\DUrole{w,w}{  }\DUrole{n,n}{str}\DUrole{w,w}{  }\DUrole{o,o}{=}\DUrole{w,w}{  }\DUrole{default_value}{\textquotesingle{}current\textquotesingle{}}}}{{ $\rightarrow$ None}}
\pysigstopsignatures
\sphinxAtStartPar
This method aborts the session while providing a reason for doing so.

\sphinxAtStartPar
It sets the Session’s \sphinxcode{\sphinxupquote{state}} flag to “aborted” in the instance
attribute as well as the \sphinxcode{\sphinxupquote{parameter}} attribute. After this, it writes a
Session Event “SEEV \%reason\%: SESSION \%obsID\% ABORTED” to the protocol
and updates the .aol parameter log.
\begin{quote}\begin{description}
\sphinxlineitem{Parameters}\begin{itemize}
\item {} 
\sphinxAtStartPar
\sphinxstyleliteralstrong{\sphinxupquote{reason}} (\sphinxcode{\sphinxupquote{str}}) \textendash{} The reason why this session had to be aborted.

\item {} 
\sphinxAtStartPar
\sphinxstyleliteralstrong{\sphinxupquote{time}} (\sphinxcode{\sphinxupquote{str}}, optional) \textendash{} An ISO 8601 conform string of the UTC datetime you want your
observation to be aborted at. Can also be “current”, in which case the
current UTC datetime will be used, defaults to “current”.

\end{itemize}

\end{description}\end{quote}

\end{fulllineitems}

\index{end() (aop.aop.Session method)@\spxentry{end()}\spxextra{aop.aop.Session method}}

\begin{fulllineitems}
\phantomsection\label{\detokenize{autoapi/aop/aop/index:aop.aop.Session.end}}
\pysigstartsignatures
\pysiglinewithargsret{\sphinxbfcode{\sphinxupquote{end}}}{\sphinxparam{\DUrole{n,n}{time}\DUrole{p,p}{:}\DUrole{w,w}{  }\DUrole{n,n}{str}\DUrole{w,w}{  }\DUrole{o,o}{=}\DUrole{w,w}{  }\DUrole{default_value}{\textquotesingle{}current\textquotesingle{}}}}{{ $\rightarrow$ None}}
\pysigstopsignatures
\sphinxAtStartPar
This method is called to end the observing session.

\sphinxAtStartPar
It sets the Session’s \sphinxcode{\sphinxupquote{state}} flag to “ended” in the instance
attribute as well as the \sphinxcode{\sphinxupquote{parameter}} attribute. After this, it writes a
Session Event “SEEV SESSION \%obsID\% ENDED” to the protocol and updates
the .aol parameter log.
\begin{quote}\begin{description}
\sphinxlineitem{Parameters}
\sphinxAtStartPar
\sphinxstyleliteralstrong{\sphinxupquote{time}} (\sphinxcode{\sphinxupquote{str}}, optional) \textendash{} An ISO 8601 conform string of the UTC datetime you want your
observation to be ended at. Can also be “current”, in which case the
current UTC datetime will be used, defaults to “current”.

\end{description}\end{quote}

\end{fulllineitems}

\index{comment() (aop.aop.Session method)@\spxentry{comment()}\spxextra{aop.aop.Session method}}

\begin{fulllineitems}
\phantomsection\label{\detokenize{autoapi/aop/aop/index:aop.aop.Session.comment}}
\pysigstartsignatures
\pysiglinewithargsret{\sphinxbfcode{\sphinxupquote{comment}}}{\sphinxparam{\DUrole{n,n}{comment}\DUrole{p,p}{:}\DUrole{w,w}{  }\DUrole{n,n}{str}}, \sphinxparam{\DUrole{n,n}{time}\DUrole{p,p}{:}\DUrole{w,w}{  }\DUrole{n,n}{str}\DUrole{w,w}{  }\DUrole{o,o}{=}\DUrole{w,w}{  }\DUrole{default_value}{\textquotesingle{}current\textquotesingle{}}}}{{ $\rightarrow$ None}}
\pysigstopsignatures
\sphinxAtStartPar
This method adds an observer’s comment to the protocol.

\sphinxAtStartPar
It writes an Observer’s Comment “OBSC \%comment\%” to the protocol, where
the AOP argument is whatever string is passed as the ‘comment’ argument
to the method verbatim.
\begin{quote}\begin{description}
\sphinxlineitem{Parameters}\begin{itemize}
\item {} 
\sphinxAtStartPar
\sphinxstyleliteralstrong{\sphinxupquote{comment}} (\sphinxcode{\sphinxupquote{str}}) \textendash{} Whatever you want your comment to read in the protocol.

\item {} 
\sphinxAtStartPar
\sphinxstyleliteralstrong{\sphinxupquote{time}} (\sphinxcode{\sphinxupquote{str}}, optional) \textendash{} An ISO 8601 conform string of the UTC datetime you want your
comment to be added at. Can also be “current”, in which case the
current UTC datetime will be used, defaults to “current”.

\end{itemize}

\end{description}\end{quote}

\end{fulllineitems}

\index{issue() (aop.aop.Session method)@\spxentry{issue()}\spxextra{aop.aop.Session method}}

\begin{fulllineitems}
\phantomsection\label{\detokenize{autoapi/aop/aop/index:aop.aop.Session.issue}}
\pysigstartsignatures
\pysiglinewithargsret{\sphinxbfcode{\sphinxupquote{issue}}}{\sphinxparam{\DUrole{n,n}{severity}\DUrole{p,p}{:}\DUrole{w,w}{  }\DUrole{n,n}{str}}, \sphinxparam{\DUrole{n,n}{message}\DUrole{p,p}{:}\DUrole{w,w}{  }\DUrole{n,n}{str}}, \sphinxparam{\DUrole{n,n}{time}\DUrole{p,p}{:}\DUrole{w,w}{  }\DUrole{n,n}{str}\DUrole{w,w}{  }\DUrole{o,o}{=}\DUrole{w,w}{  }\DUrole{default_value}{\textquotesingle{}current\textquotesingle{}}}}{{ $\rightarrow$ None}}
\pysigstopsignatures
\sphinxAtStartPar
This method is called to report an issue to the protocol.

\sphinxAtStartPar
It evaluates the \sphinxcode{\sphinxupquote{severity}} argument and after that writes the
corresponding ISSU (Issue) to the protocol:
\begin{itemize}
\item {} 
\sphinxAtStartPar
“potential”/”p”: “ISSU Potential Issue: \%message\%”

\item {} 
\sphinxAtStartPar
“normal”/”n”: “ISSU Normal Issue: \%message\%”

\item {} 
\sphinxAtStartPar
“major”/”m”: “ISSU Major Issue: \%message\%”

\end{itemize}
\begin{quote}\begin{description}
\sphinxlineitem{Parameters}\begin{itemize}
\item {} 
\sphinxAtStartPar
\sphinxstyleliteralstrong{\sphinxupquote{severity}} (\sphinxcode{\sphinxupquote{str}}) \textendash{} 
\sphinxAtStartPar
An indicator of the issue’s severity. It has to be one of the
following strings:
\begin{itemize}
\item {} 
\sphinxAtStartPar
”potential”

\item {} 
\sphinxAtStartPar
”p”

\item {} 
\sphinxAtStartPar
”normal”

\item {} 
\sphinxAtStartPar
”n”

\item {} 
\sphinxAtStartPar
”major”

\item {} 
\sphinxAtStartPar
”m”.

\end{itemize}


\item {} 
\sphinxAtStartPar
\sphinxstyleliteralstrong{\sphinxupquote{message}} (\sphinxcode{\sphinxupquote{str}}) \textendash{} A short description of the issue that is printed to the protocol
verbatim.

\item {} 
\sphinxAtStartPar
\sphinxstyleliteralstrong{\sphinxupquote{time}} (\sphinxcode{\sphinxupquote{str}}, optional) \textendash{} An ISO 8601 conform string of the UTC datetime you want your
issue to be reported at. Can also be “current”, in which case the
current UTC datetime will be used, defaults to “current”.

\end{itemize}

\sphinxlineitem{Raises}
\sphinxAtStartPar
\sphinxstyleliteralstrong{\sphinxupquote{ValueError}} \textendash{} 
\sphinxAtStartPar
If an improper value is passed in the ‘severity’ argument, that is
anything different from:
\begin{itemize}
\item {} 
\sphinxAtStartPar
”potential”

\item {} 
\sphinxAtStartPar
”p”

\item {} 
\sphinxAtStartPar
”normal”

\item {} 
\sphinxAtStartPar
”n”

\item {} 
\sphinxAtStartPar
”major”

\item {} 
\sphinxAtStartPar
”m”.

\end{itemize}


\end{description}\end{quote}

\end{fulllineitems}

\index{point\_to\_name() (aop.aop.Session method)@\spxentry{point\_to\_name()}\spxextra{aop.aop.Session method}}

\begin{fulllineitems}
\phantomsection\label{\detokenize{autoapi/aop/aop/index:aop.aop.Session.point_to_name}}
\pysigstartsignatures
\pysiglinewithargsret{\sphinxbfcode{\sphinxupquote{point\_to\_name}}}{\sphinxparam{\DUrole{n,n}{targets}\DUrole{p,p}{:}\DUrole{w,w}{  }\DUrole{n,n}{list}}, \sphinxparam{\DUrole{n,n}{time}\DUrole{p,p}{:}\DUrole{w,w}{  }\DUrole{n,n}{str}\DUrole{w,w}{  }\DUrole{o,o}{=}\DUrole{w,w}{  }\DUrole{default_value}{\textquotesingle{}current\textquotesingle{}}}}{{ $\rightarrow$ None}}
\pysigstopsignatures
\sphinxAtStartPar
This method indicates the pointing to a target identified by name.

\sphinxAtStartPar
It starts by evaluating the ‘targets’ argument provided to the method
and constructing a comma\sphinxhyphen{}separated list of targets, which is then
written in the AOP argument position in the Pointing “POIN Pointing at
target(s): \%list of targets\%” that is being written to the protocol.
\begin{quote}\begin{description}
\sphinxlineitem{Parameters}\begin{itemize}
\item {} 
\sphinxAtStartPar
\sphinxstyleliteralstrong{\sphinxupquote{targets}} (\sphinxcode{\sphinxupquote{list{[}any{]}}}) \textendash{} A list object that contains whatever objects represent the targets,
most likely strings, but it could be any other object.

\item {} 
\sphinxAtStartPar
\sphinxstyleliteralstrong{\sphinxupquote{time}} (\sphinxcode{\sphinxupquote{str}}, optional) \textendash{} An ISO 8601 conform string of the UTC datetime you want your
pointing to be reported at. Can also be “current”, in which case the
current UTC datetime will be used, defaults to “current”.

\end{itemize}

\sphinxlineitem{Raises}
\sphinxAtStartPar
\sphinxstyleliteralstrong{\sphinxupquote{TypeError}} \textendash{} If the \sphinxcode{\sphinxupquote{targets}} argument is not of type \sphinxcode{\sphinxupquote{list}}.

\end{description}\end{quote}

\end{fulllineitems}

\index{point\_to\_coords() (aop.aop.Session method)@\spxentry{point\_to\_coords()}\spxextra{aop.aop.Session method}}

\begin{fulllineitems}
\phantomsection\label{\detokenize{autoapi/aop/aop/index:aop.aop.Session.point_to_coords}}
\pysigstartsignatures
\pysiglinewithargsret{\sphinxbfcode{\sphinxupquote{point\_to\_coords}}}{\sphinxparam{\DUrole{n,n}{ra}\DUrole{p,p}{:}\DUrole{w,w}{  }\DUrole{n,n}{float}}, \sphinxparam{\DUrole{n,n}{dec}\DUrole{p,p}{:}\DUrole{w,w}{  }\DUrole{n,n}{float}}, \sphinxparam{\DUrole{n,n}{time}\DUrole{p,p}{:}\DUrole{w,w}{  }\DUrole{n,n}{str}\DUrole{w,w}{  }\DUrole{o,o}{=}\DUrole{w,w}{  }\DUrole{default_value}{\textquotesingle{}current\textquotesingle{}}}}{{ $\rightarrow$ None}}
\pysigstopsignatures
\sphinxAtStartPar
This method indicates the pointing to ICRS coordinates.

\sphinxAtStartPar
It takes a ‘ra’ argument for right ascension and a ‘dec’ argument for
declination. After that, a Pointing “Pointing at coordinates: R.A.:
\%ra\% Dec.: \%dec\%” is written to the protocol. Provide decimal degrees
for declination and decimal hours for right ascension.
\begin{quote}\begin{description}
\sphinxlineitem{Parameters}\begin{itemize}
\item {} 
\sphinxAtStartPar
\sphinxstyleliteralstrong{\sphinxupquote{ra}} (\sphinxcode{\sphinxupquote{float}}) \textendash{} Right ascension in the ICRS coordinate framework.

\item {} 
\sphinxAtStartPar
\sphinxstyleliteralstrong{\sphinxupquote{dec}} (\sphinxcode{\sphinxupquote{float}}) \textendash{} Declination in the ICRS coordinate framework.

\item {} 
\sphinxAtStartPar
\sphinxstyleliteralstrong{\sphinxupquote{time}} (\sphinxcode{\sphinxupquote{str}}, optional) \textendash{} An ISO 8601 conform string of the UTC datetime you want your
pointing to be reported at. Can also be “current”, in which case the
current UTC datetime will be used, defaults to “current”.

\end{itemize}

\sphinxlineitem{Raises}\begin{itemize}
\item {} 
\sphinxAtStartPar
\sphinxstyleliteralstrong{\sphinxupquote{TypeError}} \textendash{} If ‘ra’ or ‘dec’ is not of type ‘float’.

\item {} 
\sphinxAtStartPar
\sphinxstyleliteralstrong{\sphinxupquote{ValueError}} \textendash{} If ‘ra’ is not 0.0h \textless{}= ‘ra’ \textless{} 24.0h.

\item {} 
\sphinxAtStartPar
\sphinxstyleliteralstrong{\sphinxupquote{ValueError}} \textendash{} If ‘dec’ is not \sphinxhyphen{}90.0° \textless{}= ‘dec’ \textless{}= 90.0°.

\end{itemize}

\end{description}\end{quote}

\end{fulllineitems}

\index{take\_frame() (aop.aop.Session method)@\spxentry{take\_frame()}\spxextra{aop.aop.Session method}}

\begin{fulllineitems}
\phantomsection\label{\detokenize{autoapi/aop/aop/index:aop.aop.Session.take_frame}}
\pysigstartsignatures
\pysiglinewithargsret{\sphinxbfcode{\sphinxupquote{take\_frame}}}{\sphinxparam{\DUrole{n,n}{n}\DUrole{p,p}{:}\DUrole{w,w}{  }\DUrole{n,n}{int}}, \sphinxparam{\DUrole{n,n}{ftype}\DUrole{p,p}{:}\DUrole{w,w}{  }\DUrole{n,n}{str}}, \sphinxparam{\DUrole{n,n}{iso}\DUrole{p,p}{:}\DUrole{w,w}{  }\DUrole{n,n}{int}}, \sphinxparam{\DUrole{n,n}{expt}\DUrole{p,p}{:}\DUrole{w,w}{  }\DUrole{n,n}{float}}, \sphinxparam{\DUrole{n,n}{ap}\DUrole{p,p}{:}\DUrole{w,w}{  }\DUrole{n,n}{float}}, \sphinxparam{\DUrole{n,n}{time}\DUrole{p,p}{:}\DUrole{w,w}{  }\DUrole{n,n}{str}\DUrole{w,w}{  }\DUrole{o,o}{=}\DUrole{w,w}{  }\DUrole{default_value}{\textquotesingle{}current\textquotesingle{}}}}{{ $\rightarrow$ None}}
\pysigstopsignatures
\sphinxAtStartPar
This method reports the taking of one or more frame(s) of the same target and the camera settings used.

\sphinxAtStartPar
It evaluates the type of frame provided in the \sphinxcode{\sphinxupquote{ftype}} parameter to
arrive at a string to be written to the protocol as frame type:
\begin{itemize}
\item {} 
\sphinxAtStartPar
“science frame”/”science”/”sf”/”s”: “science”

\item {} 
\sphinxAtStartPar
“dark frame”/”dark”/”df”/”d”: “dark”

\item {} 
\sphinxAtStartPar
“flat frame”/”flat”/”ff”/”f”: “flat”

\item {} 
\sphinxAtStartPar
“bias frame”/”bias”/”bf”/”b”: “bias”

\item {} 
\sphinxAtStartPar
“pointing frame”/”pointing”/”pf”/”p”: “pointing”

\end{itemize}

\sphinxAtStartPar
Finally, the method writes a Frame “FRAM \%n\% \%frame type\% frame(s)
taken with settings: Exp.t.: \%expt\%s, Ap.: f/\%ap\%, ISO: \%ISO\%” to the
protocol.
\begin{quote}\begin{description}
\sphinxlineitem{Parameters}\begin{itemize}
\item {} 
\sphinxAtStartPar
\sphinxstyleliteralstrong{\sphinxupquote{n}} (\sphinxcode{\sphinxupquote{int}}) \textendash{} Number of frames of the specified frame type and settings that were
taken of the same target.

\item {} 
\sphinxAtStartPar
\sphinxstyleliteralstrong{\sphinxupquote{ftype}} (\sphinxcode{\sphinxupquote{str}}) \textendash{} Type of frame, ftype must not be anything other than:
* “science frame”
* “science”
* “sf”
* “s”
* “dark frame”
* “dark”
* “df”
* “d”
* “flat frame”
* “flat”
* “ff”
* “f”
* “bias frame”
* “bias”
* “bf”
* “b”
* “pointing frame”
* “pointing”
* “pf”
* “p”.

\item {} 
\sphinxAtStartPar
\sphinxstyleliteralstrong{\sphinxupquote{iso}} (\sphinxcode{\sphinxupquote{int}}) \textendash{} ISO setting that was used for the frame(s).

\item {} 
\sphinxAtStartPar
\sphinxstyleliteralstrong{\sphinxupquote{ap}} (\sphinxcode{\sphinxupquote{float}}) \textendash{} The denominator of the aperture setting that was used for the
frame(s). For example, if f/5.6 was used, provide ap=5.6 to the
method.

\item {} 
\sphinxAtStartPar
\sphinxstyleliteralstrong{\sphinxupquote{expt}} (\sphinxcode{\sphinxupquote{float}}) \textendash{} Exposure time that was used for the frame(s), given in seconds.

\item {} 
\sphinxAtStartPar
\sphinxstyleliteralstrong{\sphinxupquote{time}} \textendash{} An ISO 8601 conform string of the UTC datetime you want your
frame(s) to be reported at. Can also be “current”, in which case the
current UTC datetime will be used, defaults to “current”.

\end{itemize}

\sphinxlineitem{Raises}\begin{itemize}
\item {} 
\sphinxAtStartPar
\sphinxstyleliteralstrong{\sphinxupquote{TypeError}} \textendash{} If one of the parameters is not of the required type:
* n: int
* ftype: str
* expt: float
* ap: float
* iso: int

\item {} 
\sphinxAtStartPar
\sphinxstyleliteralstrong{\sphinxupquote{ValueError}} \textendash{} 
\sphinxAtStartPar
If an improper value is passed in the ‘ftype’ argument, that is
anything other than:
\begin{itemize}
\item {} 
\sphinxAtStartPar
”science frame”

\item {} 
\sphinxAtStartPar
”science”

\item {} 
\sphinxAtStartPar
”sf”

\item {} 
\sphinxAtStartPar
”s”

\item {} 
\sphinxAtStartPar
”dark frame”

\item {} 
\sphinxAtStartPar
”dark”

\item {} 
\sphinxAtStartPar
”df”

\item {} 
\sphinxAtStartPar
”d”

\item {} 
\sphinxAtStartPar
”flat frame”

\item {} 
\sphinxAtStartPar
”flat”

\item {} 
\sphinxAtStartPar
”ff”

\item {} 
\sphinxAtStartPar
”f”

\item {} 
\sphinxAtStartPar
”bias frame”

\item {} 
\sphinxAtStartPar
”bias”

\item {} 
\sphinxAtStartPar
”bf”

\item {} 
\sphinxAtStartPar
”b”

\item {} 
\sphinxAtStartPar
”pointing frame”

\item {} 
\sphinxAtStartPar
”pointing”

\item {} 
\sphinxAtStartPar
”pf”

\item {} 
\sphinxAtStartPar
”p”.

\end{itemize}


\end{itemize}

\end{description}\end{quote}

\end{fulllineitems}

\index{condition\_report() (aop.aop.Session method)@\spxentry{condition\_report()}\spxextra{aop.aop.Session method}}

\begin{fulllineitems}
\phantomsection\label{\detokenize{autoapi/aop/aop/index:aop.aop.Session.condition_report}}
\pysigstartsignatures
\pysiglinewithargsret{\sphinxbfcode{\sphinxupquote{condition\_report}}}{\sphinxparam{\DUrole{n,n}{description}\DUrole{p,p}{:}\DUrole{w,w}{  }\DUrole{n,n}{str}\DUrole{w,w}{  }\DUrole{o,o}{=}\DUrole{w,w}{  }\DUrole{default_value}{None}}, \sphinxparam{\DUrole{n,n}{temp}\DUrole{p,p}{:}\DUrole{w,w}{  }\DUrole{n,n}{float}\DUrole{w,w}{  }\DUrole{o,o}{=}\DUrole{w,w}{  }\DUrole{default_value}{None}}, \sphinxparam{\DUrole{n,n}{pressure}\DUrole{p,p}{:}\DUrole{w,w}{  }\DUrole{n,n}{float}\DUrole{w,w}{  }\DUrole{o,o}{=}\DUrole{w,w}{  }\DUrole{default_value}{None}}, \sphinxparam{\DUrole{n,n}{humidity}\DUrole{p,p}{:}\DUrole{w,w}{  }\DUrole{n,n}{float}\DUrole{w,w}{  }\DUrole{o,o}{=}\DUrole{w,w}{  }\DUrole{default_value}{None}}, \sphinxparam{\DUrole{n,n}{time}\DUrole{p,p}{:}\DUrole{w,w}{  }\DUrole{n,n}{str}\DUrole{w,w}{  }\DUrole{o,o}{=}\DUrole{w,w}{  }\DUrole{default_value}{\textquotesingle{}current\textquotesingle{}}}}{{ $\rightarrow$ None}}
\pysigstopsignatures
\sphinxAtStartPar
This method reports a condition description or measurement.

\sphinxAtStartPar
Every argument is optional, just pass the values for the arguments you
want to protocol. Each argument will be processed completely
separately, so a separate protocol entry will be produced for every
argument you provide. For each type of condition report, a
corresponding flag will be set as an instance attribute as well as a
parameter. This flag update will also be written to the parameter log.
In case a description is provided, a Condition Description “CDES
\%description\%” is written to the protocol, if a temperature, pressure
or humidity is provided, however, a Condition Measurement is written to
protocol for each measurement provided, respectively:
\begin{itemize}
\item {} 
\sphinxAtStartPar
“CMES Temperature: \%temp\%°C” and/or

\item {} 
\sphinxAtStartPar
“CMES Air Pressure: \%pressure\% hPa” and/or

\item {} 
\sphinxAtStartPar
“CMES Air Humidity: \%humidity\%\%”

\end{itemize}
\begin{quote}\begin{description}
\sphinxlineitem{Parameters}\begin{itemize}
\item {} 
\sphinxAtStartPar
\sphinxstyleliteralstrong{\sphinxupquote{description}} (\sphinxcode{\sphinxupquote{str}}, optional) \textendash{} A short description of every relevant element influencing the
overall observing description, but do not provide any measurements,
as these are a Condition Measurement rather than a Condition
Description, defaults to None.

\item {} 
\sphinxAtStartPar
\sphinxstyleliteralstrong{\sphinxupquote{temp}} (\sphinxcode{\sphinxupquote{float}}, optional) \textendash{} The measured temperature in °C, defaults to None.

\item {} 
\sphinxAtStartPar
\sphinxstyleliteralstrong{\sphinxupquote{pressure}} (\sphinxcode{\sphinxupquote{float}}, optional) \textendash{} The measured air pressure in hPa, defaults to None.

\item {} 
\sphinxAtStartPar
\sphinxstyleliteralstrong{\sphinxupquote{humidity}} (\sphinxcode{\sphinxupquote{float}}, optional) \textendash{} The measured air humidity in \%, defaults to None.

\item {} 
\sphinxAtStartPar
\sphinxstyleliteralstrong{\sphinxupquote{time}} (\sphinxcode{\sphinxupquote{str}}, optional) \textendash{} An ISO 8601 conform string of the UTC datetime you want your
condition update to be reported at. Can also be “current”, in which
case the current UTC datetime will be used, defaults to “current”.

\end{itemize}

\end{description}\end{quote}

\end{fulllineitems}


\end{fulllineitems}

\index{parse\_session() (in module aop.aop)@\spxentry{parse\_session()}\spxextra{in module aop.aop}}

\begin{fulllineitems}
\phantomsection\label{\detokenize{autoapi/aop/aop/index:aop.aop.parse_session}}
\pysigstartsignatures
\pysiglinewithargsret{\sphinxcode{\sphinxupquote{aop.aop.}}\sphinxbfcode{\sphinxupquote{parse\_session}}}{\sphinxparam{\DUrole{n,n}{filepath}\DUrole{p,p}{:}\DUrole{w,w}{  }\DUrole{n,n}{str}}, \sphinxparam{\DUrole{n,n}{session\_id}\DUrole{p,p}{:}\DUrole{w,w}{  }\DUrole{n,n}{str}}}{{ $\rightarrow$ {\hyperref[\detokenize{autoapi/aop/aop/index:aop.aop.Session}]{\sphinxcrossref{Session}}}}}
\pysigstopsignatures
\sphinxAtStartPar
This function parses a session from memory to a new Session object.

\sphinxAtStartPar
Provided with the filepath to the general location where the protocol and
log files are stored and an observation ID, it reads in the observation
parameters from the session’s parameter log. This information is then used
to create a new Session object, which is returned by the function. If it
has no observationID attribute, it is recreated from the function input.
\begin{quote}\begin{description}
\sphinxlineitem{Parameters}\begin{itemize}
\item {} 
\sphinxAtStartPar
\sphinxstyleliteralstrong{\sphinxupquote{filepath}} (\sphinxstyleliteralemphasis{\sphinxupquote{str}}) \textendash{} The path to the file where you expect the session directory to reside.
This is most likely equivalent to the path passed to the Session class
to create its files in, which in turn is most likely somewhere in the
installation directory of the implementing script.

\item {} 
\sphinxAtStartPar
\sphinxstyleliteralstrong{\sphinxupquote{session\_id}} (\sphinxstyleliteralemphasis{\sphinxupquote{str}}) \textendash{} The observationID of the session to be parsed.

\end{itemize}

\end{description}\end{quote}

\sphinxAtStartPar
:raises \sphinxcode{\sphinxupquote{AolNotFoundError}}: If there is no .aol file using the specified filepath and session\_id.
:raises \sphinxcode{\sphinxupquote{SessionIdDoesntExistOnFilepathError}}: If the specified session\_id is not in the filepath provided.
:raises NotADirectoryError: If the specified filepath does not constitute a directory.
\begin{quote}\begin{description}
\sphinxlineitem{Returns}
\sphinxAtStartPar
The new Session object parsed from the stored observation parameters.
For all intents and purposes, this object is equivalent to the object
whose parameters were used to parse, and you can use it to continue your
observation session or protocol just the same. Just be careful not to
run the Session.start() method again, as this would overwrite the
existing protocol instead of continuing it!

\sphinxlineitem{Return type}
\sphinxAtStartPar
{\hyperref[\detokenize{autoapi/aop/aop/index:aop.aop.Session}]{\sphinxcrossref{Session}}}

\end{description}\end{quote}

\end{fulllineitems}


\sphinxstepscope


\subsubsection{\sphinxstyleliteralintitle{\sphinxupquote{aop.tools}}}
\label{\detokenize{autoapi/aop/tools/index:module-aop.tools}}\label{\detokenize{autoapi/aop/tools/index:aop-tools}}\label{\detokenize{autoapi/aop/tools/index::doc}}\index{module@\spxentry{module}!aop.tools@\spxentry{aop.tools}}\index{aop.tools@\spxentry{aop.tools}!module@\spxentry{module}}
\sphinxAtStartPar
@author: Amélie Solveigh Hohe

\sphinxAtStartPar
This module contains auxiliary classes and functions for the aop package.


\paragraph{Module Contents}
\label{\detokenize{autoapi/aop/tools/index:module-contents}}\index{AolFileAlreadyExistsError@\spxentry{AolFileAlreadyExistsError}}

\begin{fulllineitems}
\phantomsection\label{\detokenize{autoapi/aop/tools/index:aop.tools.AolFileAlreadyExistsError}}
\pysigstartsignatures
\pysiglinewithargsret{\sphinxbfcode{\sphinxupquote{exception\DUrole{w,w}{  }}}\sphinxcode{\sphinxupquote{aop.tools.}}\sphinxbfcode{\sphinxupquote{AolFileAlreadyExistsError}}}{\sphinxparam{\DUrole{n,n}{filepath}\DUrole{p,p}{:}\DUrole{w,w}{  }\DUrole{n,n}{str}}, \sphinxparam{\DUrole{n,n}{session\_id}\DUrole{p,p}{:}\DUrole{w,w}{  }\DUrole{n,n}{str}}}{}
\pysigstopsignatures
\sphinxAtStartPar
Bases: \sphinxcode{\sphinxupquote{Exception}}

\sphinxAtStartPar
An error raised upon trying to initialize an .aol file that already exists.

\end{fulllineitems}

\index{AolNotFoundError@\spxentry{AolNotFoundError}}

\begin{fulllineitems}
\phantomsection\label{\detokenize{autoapi/aop/tools/index:aop.tools.AolNotFoundError}}
\pysigstartsignatures
\pysiglinewithargsret{\sphinxbfcode{\sphinxupquote{exception\DUrole{w,w}{  }}}\sphinxcode{\sphinxupquote{aop.tools.}}\sphinxbfcode{\sphinxupquote{AolNotFoundError}}}{\sphinxparam{\DUrole{n,n}{session\_id}\DUrole{p,p}{:}\DUrole{w,w}{  }\DUrole{n,n}{str}}}{}
\pysigstopsignatures
\sphinxAtStartPar
Bases: \sphinxcode{\sphinxupquote{Exception}}

\sphinxAtStartPar
An error raised upon trying to load an .aol file that doesn’t exist.

\end{fulllineitems}

\index{AopFileAlreadyExistsError@\spxentry{AopFileAlreadyExistsError}}

\begin{fulllineitems}
\phantomsection\label{\detokenize{autoapi/aop/tools/index:aop.tools.AopFileAlreadyExistsError}}
\pysigstartsignatures
\pysiglinewithargsret{\sphinxbfcode{\sphinxupquote{exception\DUrole{w,w}{  }}}\sphinxcode{\sphinxupquote{aop.tools.}}\sphinxbfcode{\sphinxupquote{AopFileAlreadyExistsError}}}{\sphinxparam{\DUrole{n,n}{filepath}\DUrole{p,p}{:}\DUrole{w,w}{  }\DUrole{n,n}{str}}, \sphinxparam{\DUrole{n,n}{session\_id}\DUrole{p,p}{:}\DUrole{w,w}{  }\DUrole{n,n}{str}}}{}
\pysigstopsignatures
\sphinxAtStartPar
Bases: \sphinxcode{\sphinxupquote{Exception}}

\sphinxAtStartPar
An error raised upon trying to initialize an .aop file that already exists.

\end{fulllineitems}

\index{InvalidTimeStringError@\spxentry{InvalidTimeStringError}}

\begin{fulllineitems}
\phantomsection\label{\detokenize{autoapi/aop/tools/index:aop.tools.InvalidTimeStringError}}
\pysigstartsignatures
\pysiglinewithargsret{\sphinxbfcode{\sphinxupquote{exception\DUrole{w,w}{  }}}\sphinxcode{\sphinxupquote{aop.tools.}}\sphinxbfcode{\sphinxupquote{InvalidTimeStringError}}}{\sphinxparam{\DUrole{n,n}{invalid\_string}\DUrole{p,p}{:}\DUrole{w,w}{  }\DUrole{n,n}{str}}}{}
\pysigstopsignatures
\sphinxAtStartPar
Bases: \sphinxcode{\sphinxupquote{Exception}}

\sphinxAtStartPar
An error raised upon providing a string to current\_jd’s time argument that is not interpretable as a time.

\end{fulllineitems}

\index{SessionIDDoesntExistOnFilepathError@\spxentry{SessionIDDoesntExistOnFilepathError}}

\begin{fulllineitems}
\phantomsection\label{\detokenize{autoapi/aop/tools/index:aop.tools.SessionIDDoesntExistOnFilepathError}}
\pysigstartsignatures
\pysiglinewithargsret{\sphinxbfcode{\sphinxupquote{exception\DUrole{w,w}{  }}}\sphinxcode{\sphinxupquote{aop.tools.}}\sphinxbfcode{\sphinxupquote{SessionIDDoesntExistOnFilepathError}}}{\sphinxparam{\DUrole{n,n}{invalid\_id}\DUrole{p,p}{:}\DUrole{w,w}{  }\DUrole{n,n}{str}}}{}
\pysigstopsignatures
\sphinxAtStartPar
Bases: \sphinxcode{\sphinxupquote{Exception}}

\sphinxAtStartPar
An error raised when a specified session ID could not be found on the provided filepath.

\end{fulllineitems}



\chapter{Indices and tables}
\label{\detokenize{index:indices-and-tables}}\begin{itemize}
\item {} 
\sphinxAtStartPar
\DUrole{xref,std,std-ref}{genindex}

\item {} 
\sphinxAtStartPar
\DUrole{xref,std,std-ref}{modindex}

\item {} 
\sphinxAtStartPar
\DUrole{xref,std,std-ref}{search}

\end{itemize}


\renewcommand{\indexname}{Python Module Index}
\begin{sphinxtheindex}
\let\bigletter\sphinxstyleindexlettergroup
\bigletter{a}
\item\relax\sphinxstyleindexentry{aop}\sphinxstyleindexpageref{autoapi/aop/index:\detokenize{module-aop}}
\item\relax\sphinxstyleindexentry{aop.aop}\sphinxstyleindexpageref{autoapi/aop/aop/index:\detokenize{module-aop.aop}}
\item\relax\sphinxstyleindexentry{aop.tools}\sphinxstyleindexpageref{autoapi/aop/tools/index:\detokenize{module-aop.tools}}
\end{sphinxtheindex}

\renewcommand{\indexname}{Index}
\printindex
\end{document}